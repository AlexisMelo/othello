<<<<<<< HEAD
\hspace{3em}

Finalement nous avons réussi à obtenir un jeu d’Othello fonctionnel et respectant beaucoup des exigences : affichage de l’aide, jouable en mode joueur contre joueur, standard contre une IA, et en mode tournoi avec le script proposé. Nous avons également ajouté la possibilité de lancer le jeu avec un menu, certes assez sommaire, mais qui permet de lancer une partie différemment qu’en ligne de commande.

Pour revenir sur le déroulement de ce projet nous pensons globalement que cela s’est bien déroulé même si nous avons eu quelques soucis pour le respect des délais. Il aurait été évidemment possible d’améliorer le résultat en ayant plus de rigueur dans le développement et en respectant encore plus les bonnes pratiques que nous nous sommes imposé.

Plusieurs améliorations sont possibles mais nous en avons clairement identifié certaines qui sont les suivantes : la fonction rechercherTousLesCoups est utilisée à plusieurs endroits notamment pour la validité d’un coup et pourrait être appelée une seule fois puis l’argument passé de fonctions en fonctions. Il est aussi possible d’améliorer l’affichage du plateau et l’interface utilisateur, même si le C est rarement fait pour ça il existe des bibliothèques le permettant. L’intelligence artificielle a aussi bien besoin d’être améliorée car nous n’avons pas réussi à identifier d’où venait le problème qui faisait que l’IA joue toujours quasiment instantanément même à une grande profondeur.



En conclusion ce projet a permis de découvrir ce qu’était un projet d’informatique en groupe pour certains, et d’approfondir leurs connaissances pour d’autres. Les bases de chacun en C ont été consolidées par la partie développement et les compétences en conception d’application améliorées par la partie Analyse.
=======
\begin{tabular}{|*{6}{c|}}
    \hline
   	Cycle  & Ben Ayad El Ghali  & Melo Alexis  & Mesbah Zacharia  & Saivres Jérôme  & Si Ruixu \\
    \hline
    Analyse  & 5h00  & 5h00  & 5h00  &  5h00  & 5h00  \\
    \hline
   	Conception Préliminaire & 3h00  & 3h00  & 3h00  & the rest & 1h30 \\
    \hline
   	Conception Détaillée & 3h30  & 10h00  & 6h00  & 20h00  & 2h30 \\
    \hline
   	 Developpement & 4h00  & 40h00  &  70h00 & 6h00  & 5h30 \\
    \hline
   	 Test Unitaire  &js  & 1h00  & 10h00  & 8h00 & 1h00 \\
    \hline
   	 Reunion  & 7h00  & 7h00  & 7h00  & 7h00 & 7h00 \\
    \hline
   	 Gestion de projet & -- & 10h00  & --  & -- & -- \\
    \hline
\end{tabular}
>>>>>>> 82274205c2331319642ef2ceac7451240ca07dc8

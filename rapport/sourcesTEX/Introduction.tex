\hspace{3em} Dans le cadre du cours d'algorithmique et programattion avancée en C, notre promotion a travaillé,
en groupes de 4 à 5 personnes, à la réalisation d'un jeu d'Othello, entièrement développé en C.
Nous avions pour cela l'aide de nos cours, des enseignants, et devions prendre l'initiative d'approfondir
nos connaissances via les différents moyens à notre disposition (livres, internet), lorsque cela était nécessaire.

Nous avons réalisé ce jeu en équipe, et en suivant le cycle de développement en V enseigné depuis la première
 année à l'INSA. Notre équipe est composée de 3 étudiants de STPI, un étudiant Euromed-SIC, et un étudiant intégré 
 venant de DUT Info, tous les cinq suivant la formation ASI à l'INSA de Rouen Normandie. Avec nos approches différentes, 
 et nos expériences plus ou moins importantes dans le domaine des projets informatiques, nous devions réussir à livrer 
 dans les temps les dossiers et archives demandées.

 C’est un jeu de plateau simple se jouant à deux joueurs : à chaque joueur est attribué une couleur et le but est de
 terminer la partie avec le plus de pions possible de sa propre couleur sur le plateau.

Plusieurs contraintes sont imposées pour son développement 
comme l’obligation de l’utilisation du gestionnaire de version Git et de l’outil Gitlab, le respect des 
normes de qualité concernant la documentation, la structure du code ou bien les tests. Il y a non seulement 
des enjeux techniques avec l’apprentissage d’un nouveau langage pour certains mais aussi conceptuel car
 nous devons réaliser le projet de bout en bout en respectant les étapes de développement en cycle en V. 
 A cela viennent s’ajouter des contraintes de gestion de projet de cohésion d’équipe qu’il faut savoir 
 gérer pour avoir un livrable fonctionnel.

Au niveau du jeu et de ce qui est attendu, il faut proposer à l’utilisateur final la possibilité de faire 
une partie joueur contre joueur ou bien de jouer contre une intelligence artificielle développée par nos 
soins. Un mode tournoi doit aussi être possible pour faire s’affronter deux intelligences artificielles 
grâce à un script Python disponible sur le moodle de l’INSA.

La problématique qui se pose est donc la suivante : Comment proposer une version du jeu Othello fonctionnelle 
respectant toutes les contraintes techniques imposées ainsi que celles concernant la gestion de projet ?

Ce rapport témoigne de l’ensemble de la conduite du projet depuis l’analyse jusqu’au code du projet, 
en passant par ce que nous avons décidé ensembles pour l’organisation du projet.
\subsection{Règle du jeu}
    \textbf{Plateau de jeu :}
    \begin{itemize}
        \item 64 cases (8x8)
        \item Colonne de a à h
        \item Ligne de 1 à 8 
        \item Initialiser avec deux pions noirs en e4 et d5 et deux pions blancs en d4 et e5
    \end{itemize}
    \textbf{Chaque joueur pose l'un après l'autre un pion de sa couleur}
    \begin{itemize}
        \item Obligation de capturer au moins un pion de l’adversaire
        \item Si un joueur ne peut pas capturer de pion : il passe son tour
    \end{itemize}
    \textbf{Capture de pions = retourner les pions adverses}
    \begin{itemize}
        \item Il faut placer ses pions à l'extrémité d'un alignement de pions adverses contigus terminé par un de ses propres pions (ex : B N N N N)
        \item Les alignements considérés peuvent être une colonne, une ligne, ou une diagonal
        \item Si le pion placé ferme plusieurs alignements, alors la capture des pions se fait sur tous les alignements
    \end{itemize}
    \textbf{Le jeu s'arrête quand les deux joueurs ne peuvent plus poser de pion}
    \begin{itemize}
        \item Aucun des deux joueurs ne peut jouer
        \item Le plateau ne comporte plus de case vide
    \end{itemize}
    \textbf{Score : nb de pions de sa couleur} \\ \\
    \textbf{Gagnant : le joueur ayant le score le plus grand}

\subsection{Différent mode de jeu}
    \textbf{Standard :}
    \begin{itemize}
        \item Joueur vs IA
        \item Choix de la couleur des pions que l’on donne à l’IA
    \end{itemize}
    \textbf{Tournoi :}
    \begin{itemize}
        \item Obligation de capturer au moins un pion de l’adversaire
        \item 10 secondes sur les machines des salles de TP pour jouer
        \item Communication à l’aide des entrées standards et des sorties standards du “broker”
    \end{itemize}
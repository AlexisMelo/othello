\hspace{3em}Lors de l’analyse nous imaginions avoir le TAD “Coups” sous la forme d’un ensemble, car il n’y a jamais de doublon dans une liste de Coups et les opérations des ensembles nous semblaient idéales. Cependant n'ayant pas trouvé de bibliothèque intégrant les Ensembles en C (il y en avait une qui proposait des ensembles de Strings, après réflexion nous aurions pu l’utiliser en ayant 2 fonctions qui permettent d’encoder et décoder un Coup en un String, mais faute de temps nous ne l’avons pas implémenté). 

Nous sommes alors partis sur une conception avec des Listes Chaînées où le Noeud comportait un Coup et un pointeur vers un autre Noeud. Cela semblait une bonne alternative, en plus d’apporter une optimisation au niveau de la gestion de la mémoire. Nous avons alors travaillé très longtemps avec un type Liste Chainées que nous avions écrit entièrement nous même. Malheureusement, leur manipulation s’est révélée être un calvaire à cause de nombreux bugs. Léo Pacary est alors venu à notre rescousse et nous a aidé avec un type Coups qu’il avait lui même codé et qui était infiniment plus simple et efficace, nous avons alors fait le choix de le citer dans le code source en tête de fichier et de le mentionner ici en remerciements.

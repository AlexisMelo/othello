\hspace{3em}
Pour lancer le projet et assurer son suivi nous avons organisé à plusieurs reprises des réunions avec les membres du groupe du projet pour parler de son avancement. Chaque membre décrivait ce qu’il avait effectué et ce qu’il lui restait à faire. C’était aussi l’occasion de revenir sur certains points faisant débat ou simplement des interrogations qui avaient besoin d’être répondues en groupe. De plus nous faisions parfois la review des merge request ensemble car les discussions sont parfois plus simples en face à face que par le système de commentaires proposé par Gitlab.

	Pour le découpage du travail chacun avait sa partie de l’analyse à effectuer et avait en général la tâche de l’implémenter en C par la suite. Les ensembles de fonctions à réaliser étaient découpés en unités logiques pour pas que les membres n’aient à faire le pseudo code de plusieurs fonctions sans rapport entre elles : celui désigné pour la partie pour jouer un coup s’occupait aussi de la capture des pions lors du dit-coup. Certains membres plus ou moins à l’aise avec le langage C ou l’analyse en général aidaient les autres dans leur tâche et parfois plusieurs personnes travaillaient sur une même fonction. L’utilité est d’avoir plusieurs versions possibles pour pouvoir choisir la meilleure (plus optimisée, plus élégante, plus simple à implémenter ?), ça a été le cas pour la fonction de recherche de tous les coups possibles pour une couleur où Jérôme et Alexis se sont penchés sur la question.
